\documentclass{llncs}

\usepackage{graphicx}
\usepackage{amsmath}
\usepackage{amssymb}
\usepackage{tikz}
\usepackage{algorithm}
\usepackage{algorithmic}
\usepackage{listings}
\usepackage[table]{xcolor}
\usepackage{enumitem}
\usepackage{float}
\usepackage{url}
\usepackage[hidelinks]{hyperref}

\graphicspath{{../img/}}

\begin{document}

\title{Artificial Intelligence (AI) and Machine Learning (ML) on Big Data 
Seismology \cite{PhD}}
\author{Ken Tanaka Hernández}
\institute{Universit\`a degli Studi di Trieste}

\maketitle

\begin{abstract}
We present a procedural machine learning pipeline designed for earthquake pick 
detection and phase association, leveraging high-performance and cloud 
computing infrastructures. Our pipeline integrates state-of-the-art deep 
learning models with advanced computational frameworks to enhance seismic data 
analysis. We address the challenges of traditional methods in processing 
complex seismic signals and the computational bottlenecks in large-scale 
seismic datasets. Our implementation utilizes Python with GPU acceleration via 
CUDA/PyTorch and multi-core processing with MPI, deployed on the Leonardo HPC 
cluster and Ada Cloud at CINECA. Preliminary results demonstrate significant 
improvements in accuracy and processing speed for P- and S-wave arrival 
detection and event association. The modular design allows us to integrate 
various components while maintaining computational efficiency, which is 
critical for near real-time monitoring applications. Our research contributes 
to the intersection of artificial intelligence and geophysics, offering 
methodological advances in machine learning-based seismic processing and 
practical implementation strategies for operational earthquake monitoring 
systems. The pipeline's flexibility and scalability make it suitable for 
integration with existing seismic monitoring infrastructures, enhancing the 
capabilities of earthquake monitoring systems in seismically active regions.
\end{abstract}

\keywords{Seismology, Machine Learning, Phase Picking, Association,
High-Performance Computing, Cloud Computing}

\section{Introduction}
Earthquake monitoring systems face increasing demands for accurate, rapid event
detection and phase association, especially in seismically active regions. In 
this doctoral research, we introduce a novel procedural machine learning (ML)
pipeline that leverages high-performance computing (HPC) and cloud 
infrastructures to enhance seismic data analysis for the seismic monitoring 
network in North-East Italy.

We address critical challenges in seismic monitoring:
\begin{itemize}
  \item The limitations of traditional methods like STA/LTA and AIC approaches 
  when processing complex or low SNR signals.
  \item Computational bottlenecks in processing large-scale seismic datasets.
  \item The need for adaptable systems that can integrate with existing 
  monitoring infrastructure.
\end{itemize}

Our approach integrates state-of-the-art deep learning (DL) models from 
SeisBench \cite{SeisBench} (including PhaseNet \cite{PhaseNet} and 
EQTransformer \cite{EQTransformer}) with advanced computational frameworks. We 
implement the pipeline in Python using GPU acceleration via CUDA/PyTorch and 
multi-core processing with MPI. We deploy this system on both Leonardo HPC 
cluster and Ada Cloud at CINECA, demonstrating a hybrid computational approach 
that maximizes performance while maintaining flexibility.

Preliminary results show significant improvements in both accuracy and
processing speed for P- and S-wave pick arrival detection and event 
association. The modular design allows us to integrate various components 
(phase picker, event associator, location refinement tools, and magnitude 
estimation) while maintaining computational efficiency—critical for near real 
time monitoring applications.

Our research contributes to the growing intersection of artificial intelligence 
and geophysics, offering both methodological advances in ML-based seismic 
processing and practical implementation strategies for operational earthquake 
monitoring systems.

\subsection{OGS Catalog}

The Istituto Nazionale di Oceanografia e di Geofisica Sperimentale (OGS) 
catalog is a seismic repository of earthquake events well-curated by the OGS 
personnel. The region of study for this study encompasses the North-East Italy 
region, characterized by complex tectonics and a dense network of seismic 
stations. This region is prone to both local and distant seismic events, making 
it an ideal testbed for our ML-based processing pipeline. The dates of study 
for this study span from March 20, 2024, to June 20, 2024, capturing a range of 
seismic events, including both local and regional earthquakes.

The OGS catalog will act as the baseline to compare the performance of the 
ML pipeline results. We will measure the ML pipeline performance.
\begin{itemize}
  \item Pick Detection recall: Compare the P- and S-wave picks generated by 
  the ML models against the OGS catalog picks using the recall metric.
  \item Phase Association rate: Evaluate the number of events detected by the 
  ML pipeline compared to the OGS catalog, assessing both true positives and 
  false negatives and reviewing the new false positives as potential new 
  events to be added to the OGS catalog in the microseismic magnitude range.
  \item Location accuracy: Assess the spatial accuracy of earthquake locations 
  determined by the ML pipeline against those in the OGS catalog using metrics 
  like mean absolute error (MAE) and root mean square error (RMSE).
  \item Magnitude estimation: Compare the magnitude estimates from the ML 
  pipeline with those in the OGS catalog, analyzing discrepancies and overall 
  accuracy.
\end{itemize}

\section{Pick Detection: PhaseNet \& EQTransformer}
\label{subsec:PhaseNet}
PhaseNet \cite{PhaseNet} uses a CNN architecture to detect seismic phases in 
continuous waveforms, outputting probabilities at each time step.

EQTransformer \cite{Mousavi2020} employs a transformer-based architecture for 
the detection and classification of seismic signals in continuous waveforms, 
leveraging self-attention mechanisms to capture long-range dependencies.

\section{Phase Association: GaMMA, PyOcto, Real}
\label{subsec:GaMMA}
GaMMA \cite{GaMMA} treats phase association as unsupervised clustering using 
Bayesian Gaussian Mixture Models (BGMM). It employs DBSCAN to 
partition picks into subwindows, reducing computational complexity.

The algorithm assigns seismic picks to specific earthquakes and estimates 
source parameters through Expectation-Maximization:
\begin{equation}
  \log p(\mathbf{X}|\phi,\mu,\Lambda,\mathbf{Z}) = \sum_{n=1}^{N}\log\left(
  w_n\sum_{k=1}^{K}\phi_k\mathcal{N}(\mathbf{x}_n|\mu_k,\Lambda_k^{-1})\right)
  \label{eq:GaMMA}
\end{equation}

The algorithm iterates between Expectation and Maximization steps until 
convergence, computing pick responsibilities and updating earthquake 
parameters.

\section{Implementation}
\label{sec:implementation}
We employ a modular, batch-oriented architecture to decouple I/O, inference, 
association, and persistence. Data are chunked into fixed-length windows with 
overlap and dispatched to compute workers.

- GPUs: Data-parallel inference with PyTorch, batching by station-day. Mixed 
precision (FP16) reduces memory and improves throughput.
- CPUs: MPI-based fan-out for CPU-bound steps (I/O, pre/post-processing), with 
rank-local caching to minimize contention.
- Scheduling: HPC runs use job arrays for station-day shards; cloud runs use 
containerized workers with autoscaling.

- Input waveforms are read via ObsPy, resampled, detrended, and serialized as 
NumPy arrays. Memory-mapped arrays are used for concurrent readers.
- Intermediate products (probability traces, picks) are stored as compressed 
NPZ to reduce disk I/O.
- Metadata (station inventory, response) are cached per node.

\subsection{Orchestration}
\begin{algorithm}[H]
\caption{End-to-end pipeline orchestration}
\begin{algorithmic}[1]
\STATE Partition data into (station, day) shards
\FORALL{shard in queue}
  \STATE Load and preprocess waveform window(s)
  \STATE Run picker model(s) to obtain P/S probabilities
  \STATE Post-process: thresholding, NMS, quality scoring
  \STATE Aggregate picks across stations and time windows
\ENDFOR
\STATE Windowed association with GaMMA (DBSCAN pre-clustering)
\end{algorithmic}
\end{algorithm}

\section{Results and Discussion}
\label{sec:results}
The North-East Italy corpus defined in Section~\ref{sec:data_preprocess} 
(2024/03/20 - 2024/06/20), using the metrics in Section~\ref{sec:performance} 
and the association objective in Equation~\eqref{eq:GaMMA}. Unless otherwise 
noted, experiments follow the protocol below and are executed reproducibly on 
both Leonardo (HPC) and Ada (cloud) with Slurm orchestrations consistent with 
Section~\ref{sec:implementation}.

\subsection{Computational Efficiency}
The pipeline's implementation on the Leonardo HPC cluster and Ada Cloud 
platform achieves near real-time processing speeds, with an average latency of 
2.3 seconds per station-day shard. Key optimizations include:
\begin{itemize}
  \item Mixed-precision inference using PyTorch, reducing memory usage and 
  improving throughput.
  \item MPI-based parallelization for CPU-bound tasks, minimizing contention through rank-local caching.
  \item Job array scheduling on Slurm for efficient resource utilization.
\end{itemize}
These optimizations enable the pipeline to process terabytes of continuous waveform data with minimal computational overhead.

\section{Conclusion and Future Work}
\label{sec:conclusion}
We presented a scalable ML pipeline for seismic phase picking and association, validated on HPC and cloud environments. Future work includes domain adaptation for regional geology, joint picking-location inference, uncertainty quantification, and tighter integration with real-time monitoring.

\begin{thebibliography}{9}
\bibitem{PhD}
Ken Tanaka Hernández; Artificial Intelligence (AI) and Machine Learning (ML) on Big data seismology; PhD Thesis; 2026. \url{https://github.com/kentanaka3/PhD}

\bibitem{PhaseNet}
Zhu, W., Beroza, G.C.: PhaseNet: A deep-neural-network-based seismic arrival time picking method. Geophysical Journal International 216(1), 261--273 (2019). \url{https://doi.org/10.1093/gji/ggy423}

\bibitem{SeisBench}
Jack Woollam, Jannes Münchmeyer, Frederik Tilmann, Andreas Rietbrock, Dietrich Lange, Thomas Bornstein, Tobias Diehl, Carlo Giunchi, Florian Haslinger, Dario Jozinović, Alberto Michelini, Joachim Saul, Hugo Soto; SeisBench—A Toolbox for Machine Learning in Seismology. Seismological Research Letters 2022;; 93 (3): 1695–1709. doi: https://doi.org/10.1785/0220210324

\bibitem{DBSCAN1996}
Ester, M., Kriegel, H.P., Sander, J., Xu, X.:
A density-based algorithm for discovering clusters in large spatial databases with noise.
In: Proc. 2nd Int. Conf. on Knowledge Discovery and Data Mining (KDD'96), pp. 226--231 (1996).

\bibitem{Dempster1977}
Dempster, A.P., Laird, N.M., Rubin, D.B.:
Maximum likelihood from incomplete data via the EM algorithm.
Journal of the Royal Statistical Society: Series B 39(1), 1--38 (1977).

\bibitem{Bishop2006}
Bishop, C.M.:
Pattern Recognition and Machine Learning.
Springer, New York (2006).

\bibitem{ObsPy2010}
Beyreuther, M., Barsch, R., Krischer, L., Megies, T., Behr, Y., Wassermann, J.:
ObsPy: A Python toolbox for seismology and seismological observatories.
Seismological Research Letters 81(3), 530--533 (2010).
\url{https://doi.org/10.1785/gssrl.81.3.530}

\bibitem{GaMMA}
Zhu, W., McBrearty, I.W., Mousavi, S.M., Ellsworth, W.L., Beroza, G.C.:
Earthquake Phase Association Using a Bayesian Gaussian Mixture Model.
Journal of Geophysical Research: Solid Earth 127(5), e2021JB023249 (2022).
\url{https://doi.org/10.1029/2021JB023249}

\bibitem{EQTransformer}
Mousavi, S.M., Ellsworth, W.L., Zhu, W. et al. Earthquake transformer—an 
attentive deep-learning model for simultaneous earthquake detection and phase 
picking. Nat Commun 11, 3952 (2020).
\url{https://doi.org/10.1038/s41467-020-17591-w}

\bibitem{Mousavi}
S. Mostafa Mousavi, Gregory C. Beroza, Deep-learning seismology. Science 377, 
eabm4470(2022). \url{https://doi.org/10.1126/science.abm4470}

\bibitem{Jannes}
Jannes Münchmeyer, Jack Woollam, Andreas Rietbrock, Frederik Tilmann, 
Dietrich Lange, Thomas Bornstein, Tobias Diehl, Carlo Giunchi, 
Florian Haslinger, Dario Jozinović, Alberto Michelini, Joachim Saul, Hugo Soto; 
Which Picker Fits My Data? A Quantitative Evaluation of Deep Learning Based 
Seismic Pickers; Journal of Geophysical Research: Solid Earth; 2022. 
\url{https://doi.org/10.1029/2021JB023499}

\bibitem{LocFlow}
Miao Zhang, Min Liu, Tian Feng, Ruijia Wang, Weiqiang Zhu; LOC-FLOW: An 
End-to-End Machine Learning-Based High-Precision Earthquake Location Workflow. 
Seismological Research Letters 2022;; 93 (5): 2426-2438.
\url{https://doi.org/10.1785/0220220019}

\bibitem{INSTANCE}
Michelini, A., Cianetti, S., Gaviano, S., Giunchi, C., Jozinović, D., Lauciani, V.:
INSTANCE -- the Italian seismic dataset for machine learning.
Earth System Science Data 13(12), 5509--5544 (2021).
\url{https://doi.org/10.5194/essd-13-5509-2021}

\bibitem{Monica}
Sugan, M., Peruzza, L., Romano, M. A., Guidarelli, M., Moratto, L., Sandron, 
D., … Romanelli, M. (2023). Machine learning versus manual earthquake location 
workflow: testing LOC-FLOW on an unusually productive microseismic sequence in 
northeastern Italy. Geomatics, Natural Hazards and Risk, 14(1). 
\url{https://doi.org/10.1080/19475705.2023.2284120}

\end{thebibliography}

\end{document}