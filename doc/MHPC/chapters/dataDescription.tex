\chapter{Data Description}
\label{chap:dataDescription}
This chapter provides a detailed overview of the seismic datasets used in this 
study, describes their acquisition by the North-East Italy Monitoring System, 
and outlines the data preparation and quality-control procedures. We describe 
the physical characteristics of seismic waveforms, the acquisition 
infrastructure in northeastern Italy, and the composition and structure of the 
datasets used for machine-learning model evaluation. In addition, we present 
the external ML training datasets employed to benchmark model generalization 
and cross-regional robustness. Finally, we discuss the data preprocessing 
steps implemented to ensure data quality and consistency prior to model
training and evaluation.

\section{Seismic Waveform Data}
Earthquakes originate from the abrupt release of accumulated elastic strain in 
the lithosphere, generating seismic waves that propagate through the Earth's 
interior and along its surface. These waves encode information about the source 
process, subsurface structure, and propagation path, making them fundamental 
for earthquake detection and characterization. Seismic waveform data consist of 
continuous time series recordings of ground motion captured by seismometers 
deployed at various locations. Each waveform represents the ground 
displacement, velocity, or acceleration as a function of time. The waveforms 
are typically sampled at high frequencies (e.g., 100 Hz or higher) to capture 
the rapid variations associated with seismic events. The data are stored in 
standard formats such as SEED or SAC, which include metadata about the 
recording station, instrument response, and event information.

\subsection{Types of Seismic Waves}
Seismic waves can be broadly classified into \textbf{body waves} and 
\textbf{surface waves}, each characterized by distinct propagation mechanisms 
and velocities.
\begin{itemize}
  \item Body waves are seismic waves that travel through the Earth's interior 
        and are the fastest seismic waves. Body waves are further divided into 
        two types: P waves and S waves.
        \begin{itemize}
          \item P waves or Primary waves are the fastest seismic waves and 
                travel through solids, liquids, and gases. P waves are also 
                known as compressional (longitudinal) waves because they 
                compress and expand the material through which they travel.
          \item S waves are secondary waves that are slower than P waves and 
                travel only through solids. S waves are also known as shear 
                (transversal) waves because they move the material through 
                which they travel from side to side.
        \end{itemize}
  \item Surface waves are seismic waves that travel along the Earth's surface 
        and are the slowest seismic waves. Surface waves are further divided
        into two types: Love waves and Rayleigh waves.
\end{itemize}
The velocities of these phases, strictly depend on the properties of the 
material they are traveling through, in particular the rock's density, 
compressibility and shear modulus of the material. Variations in these 
properties result in measurable differences in travel-time patterns, waveform 
shapes, and frequency content, which are essential for inversion and event 
localization.

\subsection{Data Acquisition}
The seismic waveform data are recorded using a network of seismometers deployed 
across northeastern Italy. Each instrument has their own advantages and 
disadvantages, and is used for different purposes. Seismometers are instruments 
that measure the velocity of the ground motion, while accelerometers are 
instruments that measure the acceleration of the ground motion. Each sensor 
records the ground motion in three orthogonal components: vertical (Z), 
north-south (N), and east-west (E). The data are transmitted in real-time to a 
central data acquisition system, where they are stored and processed for 
earthquake detection and analysis. The waveforms are archived in standardized 
formats such as MiniSEED or SAC, which include metadata about the recording 
station, instrument response, and event information. We utilize publicly 
available algorithms and tools such as ObsPy (\cite{ObsPy}) and SeisBench 
(\cite{SeisBench}) to access, process, and analyze the seismic waveform data.
ObsPy is an open-source Python library for seismology that provides a wide 
range of tools for processing and analyzing seismic data. ObsPy includes 
modules for downloading data from various data centers called \textbf{Mass 
Downloader}, filtering, resampling, and visualization. SeisBench is an 
open-source Python library that provides a standardized interface for accessing
and processing seismic waveform data. SeisBench includes a wide range of 
predefined datasets, as well as tools for data augmentation, preprocessing, and
model evaluation.

\section{Datasets}
\subsection{Case Study}
In this study, we use continuous waveforms recorded by 147 \ac{SMINO} and 
surrounding seismic stations located at the North-Eastern Italy region (Fig. 
\ref{fig:AdriaArray}). The \ac{SMINO} is managed by OGS also on behalf of the 
\ac{FVG} and Veneto Regions. The stations are installed in \ac{FVG}, Veneto 
(in most cases), Emilia Romagna and Lombardy. The data recorded by all the 
stations are acquired in real time at the OGS Centre for Seismological Research 
in Udine. The network also provides data to the national seismic surveillance 
system, with real-time data exchange with the Civil Protection Department and 
the National Institute of Geophysics and Volcanology (\ac{INGV}). Moreover, in 
order to improve the quality of localizations and magnitude estimation in 
border regions, the network exchanges real-time data with the seismic networks 
of Austria, Slovenia, Switzerland, and the Autonomous Provinces of Trento and 
Bolzano. The waveforms are recorded in three components (Z, N, and E, which 
represent the vertical, north-south, and east-west components of the seismic 
wave respectively), and different sampling rate (100Hz, 200Hz the most 
represented). Data are stored in MSEED format and daily files for each seismic 
station and component. The analyzed timeframe is from 20/03/2024 to 20/06/2024.
During this period, a total of 428 earthquakes were recorded by the network, 
with 7152 manually reviewed picks (P and S phases) available in the catalog. 
The dataset includes a diverse range of seismic events with magnitudes ranging 
from 0 to above 4.5, with depths ranging from shallow (a few kilometers) to 
intermediate depths (up to 70 km), providing a comprehensive representation of 
the seismic activity in northeastern Italy.
