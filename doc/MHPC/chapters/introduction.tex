\doublespacing
\label{chap:introduction}
In this thesis, we present a Procedural Machine Learning Pipeline for 
Earthquake Pick Detection and Phase Association, leveraging both \ac{HPC} and 
Cloud Computing Infrastructures to enhance the accuracy and efficiency of 
seismic data analysis.\\

Earthquakes are caused by the sudden release of energy within the Earth's 
lithosphere. This energy is released in the form of seismic waves that travel 
through the Earth's interior and surface. The lithosphere itself is a highly 
dynamic and complex system that is constantly deformed by the interplay of 
tectonic forces related to mantle convection. It consists of the crust and the
uppermost mantle and is fragmented into tectonic plates that interact along 
their boundaries, leading to geophysical phenomena such as seismic activity and 
volcanic eruptions. Earthquakes are recorded by seismometers, which are 
instruments that measure the ground motion. The identification and detection of 
seismic phases associated with earthquakes are fundamental processes for 
accurate earthquake localization and subsequent seismological analysis 
(\cite{lay1995modern}). Over the years, several automated methods have been 
developed to identify the arrival times of primary (P) and secondary (S) waves 
over the past decades, such as the \ac{STA/LTA} method (\cite{STA/LTA}) and the 
\ac{AIC} approach (\cite{LEONARD1999247}). These methods have significantly 
advanced our capability to detect P and S seismic phases associated with 
earthquakes. However, they also have some limitations, \textit{e.g.} their 
performance often diminishes when analyzing complex or low \ac{SNR} signals. 
Furthermore, localization of earthquakes often relies on a supervised expert 
revision of the seismic pick, which can be time-consuming and present biases 
related to the subjective sensitivity of the human operator.\\

The inherent complexity, nonlinearity, and variability of seismic phenomena
have motivated the exploration of new methodologies that can handle large
amounts of data and uncover patterns that are difficult to detect using
conventional techniques.\\

Recently, there has been increasing interest in the use of \ac{ML} techniques 
to analyze seismic data and to improve the accuracy and efficiency of seismic 
wave phase picking and earthquake detection. \ac{ML} algorithms, particularly 
\ac{DL} models, have shown remarkable success in detecting seismic events, 
classifying seismic signals, and estimating the magnitude and location of 
earthquakes, opening new opportunities for improving seismic phase detection. 
By integrating \ac{ML} methods with large-scale computational frameworks, this 
study aims to refine seismic phase identification and association in North-East
Italy, ultimately contributing to more accurate earthquake monitoring and 
hazard assessment.\\

Among the \ac{ML}-based tools available for seismic analysis, SeisBench
\cite{SeisBench} stands out as a comprehensive and open-source platform. 
SeisBench provides a unified \ac{API} for applying state-of-the-art \ac{DL} 
models to seismic waveform data. This platform facilitates the deployment of 
pre-trained models for seismic phase picking, offering access to a variety of 
seismic datasets and model architectures tailored to detect phase arrivals. The 
ability of \ac{ML} models to recognize patterns, make inferences, and 
generalize from data makes them well-suited for the complex and data-rich field 
of seismology. Seismological datasets are often vast and multidimensional, 
which presents a significant computational "\textit{Big Data}" challenge. 
Networks of seismic stations continuously record digital waveforms that 
collectively reach storage volumes on the order of terabytes annually. The 
efficient processing of these datasets, achieved by advanced high performant 
computational techniques, including parallel processing and \acs{GPU} 
computing, has become indispensable.\\

In this thesis, we explore the application of \ac{ML} techniques to the
detection of P and S arrival times and the subsequent association of a seismic
event. Specifically, we focus on the analysis of seismic waveforms recorded by 
the \ac{SMINO} in the North-East of Italy \cite{Bragato2021}. The 
infrastructure was consituted in response to the ML 6.4 Friuli destructive 
earthquake in 1976, with the main mandate of supporting civil protection 
emergency activities.

The Master's thesis goal is to test \ac{ML} models capable of automatically
detecting earthquakes, predicting seismic wave arrival times to be used also in 
real-time. This research aims to enhance the efficiency and accuracy of seismic 
monitoring systems in North-East of Italy Fig. \ref{fig:AdriaArray}. We aim to 
deepen the understanding of seismic phenomena through digital innovation. 
Furthermore, we develop a publicly available cloud-based platform 
infrastructure provided by the \acf{OGS}. The cloud service is meant for the 
scientific community to access, analyze, train \ac{ML} models, and share the 
results of the analysis. This cloud service is provided under the \acf{TeRABIT} 
project. By integrating cutting-edge \ac{ML} methodologies and advanced 
computational resources, this well-rounded research advances the field of 
seismology and contributes to the global effort to mitigate seismic risks.\\

The thesis is organized as follows. In Chapter \ref{chap:data_description}, we 
provide an overview of the diverse datasets used in this study, including the 
seismic waveforms recorded in the North-East of Italy. We describe the data
collection process and the data cleaning procedures. In Chapter 
\ref{chap:pipeline}, we describe the data preprocessing steps, present the 
variety of methods and techniques used in the literature for seismic phase 
picking and earthquake detection. We discuss the challenges of analyzing 
seismic data and the potential of \ac{ML} techniques for addressing these 
challenges. In Chapter \ref{chap:performance}, we describe the high performance 
computing and cloud computing infrastructure provided by CINECA on LEONARDO 
Cluster and Ada Cloud, respectively, under the research project 
IscrC\_AI4Seism\_C. We provide details on the hardware and software 
configuration of the high performance computing infrastructure and discuss the 
benefits of using high performance computing. By combining high performance 
computing and cloud computing, we are able to exploit the computational 
resources of both infrastructures to reach the capabilities of urgent computing 
infrastructure, which is essential for real-time seismic monitoring. In Chapter 
\ref{chap:results}, we present the results of the \ac{ML} models on the seismic 
waveforms recorded in the North-East of Italy. We evaluate the performance of 
the models using various metrics, including the cumulative number of picks, the 
multi-class classification results, \ac{TP}, \ac{FN}, recall, and the time 
displacement. Finally, in Chapter \ref{chap:conclusions}, we summarize the main 
findings of this thesis and discuss the implications of the results for the 
field of seismology. We also provide recommendations for future research and 
discuss the potential impact of this work on the development of seismic 
monitoring systems in North-East of Italy and beyond.

\begin{comment}
Earth's lithosphere is a complex and dynamic system constantly subject to 
stresses and deformations due to the forces and rheological properties of the 
materials acting upon it. The lithosphere is composed of the crust and the 
uppermost part of the mantle and it is divided into tectonic plates that float 
on the semi-fluid asthenosphere. The motion of these plates is responsible for 
shaping Earth's surface, giving rise to mountain ranges, ocean basins, 
earthquakes and volcanic activity. The study of the lithosphere and its 
dynamics is of great importance as it helps us to understand the processes that 
take place on the Earth, allowing to predict natural hazards extit{e.g.} 
earthquakes and volcanic eruptions, which continuously reshape our planet.\\

Traditional methods consist of collecting vast amounts of data from 
seismometers, which are instruments that measure the ground motion caused by
seismic waves. Seismologists have been analyzing this data using physics-based 
models and empirical relationships to study the propagation of seismic waves 
through the Earth's crust. Historically, seismic \textbf{phase picking}, which 
identifies the arrival times of primary (P) and secondary (S) waves, and 
\textbf{earthquake detection}, which localizes earthquakes, have been performed 
manually by seismologists as a fundamental step in the process of understanding 
the dynamics of seismic events and their underlying processes.\\

The inherent complexity, nonlinearity, and variability of seismic phenomena 
have motivated the exploration of new methodologies that can handle large 
amounts of data and uncover patterns that are difficult to detect using 
conventional techniques. Several automated methods have been developed over the
past decades such as, the \ac{STA/LTA} method (Allen, 1978) and the \acf{AIC} 
approach (Leonard \& Kennett, 1999). These methods have significantly advanced 
our understanding of earthquake processes and have proven effective in many 
cases. However, they also have some limitations, \textit{e.g.} the need for 
simplifications or assumptions to the inherent complexity of the lithosphere or 
their performance often diminishes when analyzing complex or low \ac{SNR} 
signals. Furthermore, the detection and localization of earthquakes often rely 
on the manual (subjective) inspection of seismic data, which can be 
time-consuming.\\

In this thesis, we explore the application of \ac{ML} techniques to the 
detection and prediction of seismic waves. Specifically, we focus on the 
analysis of seismic waveforms recorded at the region of North-East of Italy. 
Our goal is to develop \ac{ML} models capable of automatically detecting 
earthquakes and predicting seismic wave arrival times in real-time. This 
research aligns with the objectives of the \acf{PNRR} and aims to enhance the
efficiency and accuracy of seismic monitoring systems in Italy. We aim to 
deepen the understanding of seismic phenomena through digital innovation. 
Furthermore, we develop a publicly available cloud-based platform 
infrastructure provided by the \acf{OGS}. The cloud service is meant for the 
scientific community to access, analyze, train \ac{ML} models, and share the 
results of the analysis. This cloud service is provided under the \acf{TeRABIT} 
project. By integrating cutting-edge \ac{ML} methodologies and advanced 
computational resources, this well-rounded research advances the field of 
seismology and contributes to the global effort to mitigate seismic risks.\\

\begin{comment}
After the earthquake in L'Aquila, Italy, in 2009, which caused 309 deaths and
left 65,000 people homeless, the Italian government decided to establish the
\acf{INGV} to monitor and study seismic activity in the country. The \ac{INGV}
has deployed a network of over 400 seismometers throughout Italy to detect and 
locate earthquakes and to study the propagation of seismic waves through the 
Earth's crust.
\end{comment}
