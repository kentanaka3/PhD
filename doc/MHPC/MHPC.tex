\documentclass[12pt, a4paper]{report}
\usepackage[utf8]{inputenc}
\usepackage[english]{babel}
\usepackage{csquotes}
\usepackage[T1]{fontenc}
\usepackage{geometry}
\geometry{a4paper, left=10mm, right=10mm, top=20mm, bottom=17mm,
          bindingoffset=6mm, headheight=15pt}
\usepackage{helvet}
\renewcommand{\familydefault}{\sfdefault}
\usepackage{graphicx}
\graphicspath{ {../img/} }
\usepackage{amsmath, amssymb, amsfonts, amsthm}
\usepackage[utf8]{inputenc}
\usepackage[
  style=authoryear,
  backend=biber,
  maxbibnames=99,
  maxcitenames=2,
  minnames=1,
  sorting=nyt,
  natbib=true,
  url=true,
  doi=true,
  eprint=false,
  isbn=false,
]{biblatex}
\usepackage{caption}
\usepackage{subcaption}
\usepackage{multicol}
\usepackage{comment}
\usepackage{hyperref}
\usepackage{acro}
\usepackage{pdflscape}
\usepackage[none]{hyphenat}
\usepackage{float}
\usepackage{color}
\usepackage[table]{xcolor}
\usepackage{listings}
\usepackage{enumitem}
\addbibresource{lib.bib}
\sloppy
\newenvironment{absolutelynopagebreak}
  {\par\nobreak\vfil\penalty0\vfilneg
   \vtop\bgroup}
  {\par\xdef\tpd{\the\prevdepth}\egroup
   \prevdepth=\tpd}
\DeclareAcronym{ML}{short = ML, long = Machine Learning}
\DeclareAcronym{CNN}{short = CNN, long = Convolutional Neural Network}
\DeclareAcronym{RNN}{short = RNN, long = Recurrent Neural Network}
\DeclareAcronym{OGS}{short = OGS, long = Istituto Nazionale di Oceanografia e di Geofisica Sperimentale}
\DeclareAcronym{PNRR}{short = PNRR, long = Piano Nazionale di Ripresa e Resilienza}
\DeclareAcronym{TeRABIT}{short = TeRABIT, long = Terabit Network for Research and Academic Big Data in Italy}
\DeclareAcronym{INGV}{short = INGV, long = Istituto Nazionale di Geofisica e Vulcanologia}
\DeclareAcronym{INSTANCE}{short = INSTANCE, long = Italian Seismic Dataset for Machine Learning, cite=INSTANCE}
\DeclareAcronym{SCEDC}{short = SCEDC, long = Southern California Earthquake Data Center, cite=SCEDC}
\DeclareAcronym{STEAD}{short = STEAD, long = STanford EArthquake Dataset, cite=STEAD}
\DeclareAcronym{GaMMA}{short = GaMMA, long = Gaussian Mixture Model Associator, cite=GaMMA}
\DeclareAcronym{EQTransformer}{short = EQTransformer, long = Earthquake Transformer, cite=EQTransformer}
\DeclareAcronym{KNN}{short = KNN, long = K-Nearest Neighbors}
\DeclareAcronym{SVM}{short = SVM, long = Support Vector Machine}
\DeclareAcronym{RF}{short = RF, long = Random Forest}
\DeclareAcronym{LSTM}{short = LSTM, long = Long Short-Term Memory}
\DeclareAcronym{GRU}{short = GRU, long = Gated Recurrent Unit}
\DeclareAcronym{MLP}{short = MLP, long = Multilayer Perceptron}
\DeclareAcronym{PCA}{short = PCA, long = Principal Component Analysis}
\DeclareAcronym{SVD}{short = SVD, long = Singular Value Decomposition}
\DeclareAcronym{FFT}{short = FFT, long = Fast Fourier Transform}
\DeclareAcronym{STFT}{short = STFT, long = Short-Time Fourier transform}
\DeclareAcronym{FN}{short = FN, long = False Negative}
\DeclareAcronym{FP}{short = FP, long = False Positive}
\DeclareAcronym{TN}{short = TN, long = True Negative}
\DeclareAcronym{TP}{short = TP, long = True Positive}
\DeclareAcronym{F1}{short = F1, long = F1 Score}
\DeclareAcronym{SNR}{short = SNR, long = Signal-to-Noise Ratio}
\DeclareAcronym{DL}{short = DL, long = Deep Learning}
\DeclareAcronym{HPC}{short = HPC, long = High-Performance Computing}
\DeclareAcronym{MPI}{short = MPI, long = Message Passing Interface}
\DeclareAcronym{CUDA}{short = CUDA, long = Compute Unified Device Architecture}
\DeclareAcronym{API}{short = API, long = Application Programming Interface}
\DeclareAcronym{AIC}{short = AIC, long = Akaike Information Criterion}
\DeclareAcronym{STA/LTA}{short = STA/LTA, long = Short-Time Average/Long-Time Average}
\DeclareAcronym{GPU}{short = GPU, long = Graphics Processing Unit}
\DeclareAcronym{CPU}{short = CPU, long = Central Processing Unit}
\DeclareAcronym{RAM}{short = RAM, long = Random Access Memory}
\DeclareAcronym{SSD}{short = SSD, long = Solid State Drive}
\DeclareAcronym{HDD}{short = HDD, long = Hard Disk Drive}
\DeclareAcronym{CM}{short = CM, long = Confusion Matrix}
\DeclareAcronym{SMINO}{short = SMINO, long = Northeast Italy Monitoring System}
\DeclareAcronym{FVG}{short = FVG, long = Friuli Venezia Giulia}
\usepackage[onehalfspacing]{setspace}
\begin{document}
  \begin{titlepage}
    \begin{singlespace}
      \begin{center}
        \includegraphics[height=2.1cm]{SISSA.png}\hfill
        \includegraphics[height=2.7cm]{MHPC.png}\hfill
        \includegraphics[height=2.1cm]{ICTP.jpg}
        \vspace{1.5cm}

        {\scshape\LARGE Master in High Performance Computing \par}
        \vspace{1.5cm}
        {\huge\bfseries Procedural Machine Learning Pipeline for Earthquake Pick 
        Detection and\\ Phase Association using \par High-Performance and 
        Cloud Computing infrastructures \par}
      \end{center}
      \vspace{2cm}
      \vfill
      \textit{Supervisor(s)}:\par
      Dr. \textsc{Monica Sugan},\par
      Dr. \textsc{Serafina Di Gioia}
      \vspace{1cm}
      \begin{flushright}
        \textit{Candidate}:\par Ken \textsc{Tanaka Hernández}
      \end{flushright}
      \vfill
      \centering
      {\large\textsc{$10^{\mathrm{th}}$ edition}\\[0.1cm]2023 -- 2024}
      \vfill
      \includegraphics[height=2cm]{MHPC.png}
    \end{singlespace}
  \end{titlepage}
  \pagenumbering{gobble}
  \chapter*{Acknowledgements}
  \label{chap:acknowledgements}
The present work is founded from the \href{https://www.ogs.it/en}{\ac{OGS}} and 
by the 
\href{https://www.italiadomani.gov.it/content/sogei-ng/it/en/home.html}{\ac{PNRR}}
project (\href{https://www.terabit-project.it/en/}{\ac{TeRABIT}} - IR0000022 - 
PNRR Missione 4, Componente 2, Investimento 3.1 CUP I53C21000370006) in the 
frame of the European Union - NextGenerationEU funding.

We acknowledge \href{https://www.cineca.it/en}{CINECA} for awarding this 
project access to the
\href{https://wiki.u-gov.it/confluence/display/SCAIUS/UG3.2.1\%3A+LEONARDO+Booster+UserGuide}{LEONARDO} 
and 
\href{https://wiki.u-gov.it/confluence/display/SCAIUS/UG3.5\%3A+ADA+Cloud+UserGuide}{Ada Cloud} 
supercomputers.
  \newpage
  \begin{abstract}
    A rapid increase in continuous seismic data volumes demands automated 
    solutions that can operate efficiently at regional scales. This work 
    evaluates an end-to-end, scalable machine-learning (ML) pipeline for 
    earthquake detection and characterization, designed for deployment in a 
    high-performance computing (HPC) environment. The pipeline integrates 
    publicly available models from the seismological community 
    (\cite{SeisBench}) and addresses the challenges of large-scale seismic 
    processing by combining advanced computational techniques with 
    state-of-the-art ML methods. It is configured to process seismic data 
    collected by the Northeast Italy Monitoring System (\cite{SMINO}), managed 
    by the Istituto Nazionale di Oceanografia e di Geofisica Sperimentale - 
    OGS, as well as data from other Italian and international seismic networks 
    in neighboring region and countries. This enables a comprehensive regional 
    analysis of seismic activity in northeastern Italy, allowing OGS to support 
    civil protection by issuing alerts when earthquakes occur and providing 
    rapid event solutions along with preliminary estimates of ground shaking.

    The workflow includes specialized modules for phase picking (PhaseNet, 
    \cite{PhaseNet}; EQTransformer, \cite{EQTransformer}), event association 
    (GaMMA, \cite{GaMMA}; PyOcto, \cite{PyOcto}), hypocenter localization 
    (NonLinLoc, \cite{NonLinLoc}), and magnitude estimation 
    (@Jannes, can you  please add the proper reference for your code?). 
    Deep-learning phase pickers for NE Italy are extensively evaluated using 
    diverse training datasets, including INSTANCE (\cite{INSTANCE}), STEAD 
    (\cite{STEAD}), SCEDC (\cite{SCEDC}), and the original PhaseNet training 
    dataset, to determine the most effective combinations and ensure robust 
    detection across the region. To rigorously validate the pipeline, a 
    physically informed evaluation framework is introduced, based on a 
    domain-specific confusion matrix that defines “matched,” “missed,” and 
    “proposed” detections using strict criteria, including temporal tolerances 
    of 0.5 s for picks and 1.5 s for events, and spatial constraints of 3 km 
    for event locations. A bipartite optimization scheme based on weighted 
    similarity scores is used to resolve ambiguities and ensure accurate 
    one-to-one matching with the OGS reference catalogs (e.g. 
    \cite{RSINO2020, RSINO2021}; \cite{Brondi2023, Brondi2025}). 
    Performance is quantified through recall, complemented by detailed analyses 
    of timing, spatial, and magnitude residuals to characterize systematic 
    biases and uncertainties. Supported by a HPC infrastructure, this 
    methodology establishes a robust, transparent, and reproducible standard 
    for benchmarking automated seismic monitoring systems in modern, 
    data-intensive seismology.

    Results show that the pipeline achieves high fidelity in detecting P- and 
    S-wave arrivals and in associating events, substantially outperforming 
    traditional CPU-only implementations in both speed and scalability. The HPC 
    architecture provides enhanced operational flexibility, enabling rapid 
    near-real-time analysis as well as large-scale historical reprocessing on 
    the same platform. Overall, the proposed framework offers a practical 
    blueprint for next-generation regional earthquake monitoring in 
    environments characterized by continuously growing data volumes.

  \end{abstract}
  \newpage
  \pagenumbering{roman}
  \tableofcontents
  \begin{singlespace}
    \listoffigures
    \listoftables
    \printacronyms
  \end{singlespace}
  \chapter{Introduction}
  \pagenumbering{arabic}
  In this thesis, we present a procedural \ac{ML} pipeline for
  earthquake pick detection and phase association designed to operate across 
  \ac{HPC} infrastructures. The objective is to enhance 
  the accuracy, reproducibility, and operational readiness of seismic data 
  processing workflows in the context of regional seismic monitoring. The 
  motivation for this work arises from the increasing volume and complexity of 
  continuous seismic waveform data recorded by dense seismic networks 
  worldwide, which needs an automated, scalable solution for earthquake 
  detection and characterization.

  Earthquakes originate from the abrupt release of accumulated elastic strain 
  within the Earth's lithosphere, producing seismic waves that propagate 
  through both the interior and along the surface. The lithosphere itself is a 
  mechanically heterogeneous and dynamically evolving system, continuously 
  shaped by the interaction of tectonic plates driven by mantle convection. 
  These interactions give rise to faulting, seismic activity, and other 
  geodynamic phenomena. Seismometers are deployed on the surface to convert 
  ground motion into continuous digital observations, enabling the extraction 
  of diagnostic seismic phases (primarily P and S waves) that form the basis 
  for earthquake location, magnitude estimation, and subsequent seismological 
  analyses (\cite{lay1995modern}).

  Classical automated picking procedures such as the short-term average to 
  long-term average \ac{STA/LTA} method (\cite{STA/LTA}) and techniques based 
  on the \ac{AIC} (\cite{LEONARD1999247}) provided foundational advancements in 
  large-scale phase detection. However, their performance often deteriorates in 
  low \ac{SNR} environments, or in the presence of complex wavefields commonly 
  observed in tectonically active regions. Moreover, earthquake catalogs still 
  rely on manual analyst revision, which introduces subjectivity and limits 
  scalability when processing vast seismic archives. These limitations have 
  motivated the search for more robust and generalizable detection strategies 
  capable of handling heterogeneous, high-volume waveform datasets.

  The increasing complexity and nonlinearity of seismic phenomena, together 
  with the unprecedented growth of available continuous waveform archives, have 
  stimulated significant interest in applying \ac{ML} methodologies to 
  regional and global seismic monitoring. Recent advances in \ac{DL} 
  have demonstrated compelling performance in phase picking, event detection, 
  noise discrimination, and magnitude estimation, offering a transformative 
  pathway toward fully automated seismic processing pipelines. In particular, 
  ML algorithms excel at identifying subtle patterns embedded in large 
  datasets, discovering latent structure, and generalizing across stations, 
  networks, and tectonic settings.

  Among modern ML frameworks, SeisBench \cite{SeisBench} provides a unified and 
  extensible platform for deploying advanced DL models on seismic waveform 
  data. It offers access to a diverse suite of pre-trained models, standardized 
  APIs, and curated datasets, enabling systematic testing and benchmarking of 
  model performance across different seismic contexts. At the same time, the 
  evergrowing volume of data recorded by dense seismic networks, often reaching 
  multiple terabytes per year, requires computational strategies that leverage 
  parallel processing, GPU acceleration, and distributed data management. Such 
  techniques are essential for enabling both large-scale retrospective 
  reprocessing and near-real-time earthquake monitoring.

  Against this backdrop, the present thesis investigates the application of ML 
  techniques to automated P- and S-phase detection and subsequent event 
  association using continuous seismic waveforms recorded by the North-East 
  Italy Seismic Network (SMINO) \cite{Bragato2021}. The establishment of this 
  network was motivated by the ML 6.4 Friuli earthquake of 1976, and it 
  continues to serve as a fundamental instrument for civil protection activities 
  in the region. By combining region-specific waveform characteristics with 
  state-of-the-art ML models, this thesis aims to improve the timeliness and 
  reliability of seismic monitoring in the northeastern Italian region.

  The remainder of this thesis is structured as follows.

  Chapter \ref{chap:dataDescription} provides a detailed overview of the 
  seismic datasets used in this study, describes their acquisition by the 
  North-East Italy Monitoring System, and outlines the data preparation and 
  quality-control procedures.

  Chapter \ref{chap:pipeline} introduces the complete preprocessing workflow 
  and summarizes the state of the art in seismic phase picking and earthquake 
  detection. It discusses the inherent challenges posed by complex waveform 
  characteristics and highlights how ML-based approaches can address these 
  limitations.

  Chapter \ref{chap:performance} presents the HPC and cloud-computing 
  environments used in this research, including the LEONARDO Cluster at CINECA 
  and the Ada Cloud platform. The chapter details the hardware/software 
  configurations, resource management strategies, and performance benefits 
  associated with heterogeneous computational architectures. It also 
  illustrates how the integration of HPC and cloud systems facilitates 
  near-real-time processing capabilities essential for operational seismic 
  monitoring.

  Chapter \ref{chap:results} evaluates the performance of the ML models on 
  regional seismic data using a suite of quantitative metrics, including 
  cumulative pick counts, multi-class confusion matrices, true positives, false 
  negatives, recall, and time residual analyses.

  Finally, Chapter \ref{chap:conclusions} summarizes the principal findings, 
  discusses the implications for automated regional seismic monitoring, and 
  outlines future directions for research, including the potential integration 
  of advanced associators, catalog-building methodologies, and real-time hazard 
  assessment pipelines.

  Through the combined use of cutting-edge ML methodologies, modern 
  computational infrastructures, and a regionally focused case study, this 
  thesis contributes to ongoing efforts to modernize seismic monitoring and to 
  enhance the scientific understanding of seismic processes in northeastern 
  Italy and beyond.
  \chapter{Data Description}
\label{chap:dataDescription}
This chapter provides a detailed overview of the seismic datasets used in this 
study, describes their acquisition by the North-East Italy Monitoring System, 
and outlines the data preparation and quality-control procedures. We describe 
the physical characteristics of seismic waveforms, the acquisition 
infrastructure in northeastern Italy, and the composition and structure of the 
datasets used for machine-learning model evaluation. In addition, we present 
the external ML training datasets employed to benchmark model generalization 
and cross-regional robustness. Finally, we discuss the data preprocessing 
steps implemented to ensure data quality and consistency prior to model
training and evaluation.

\section{Seismic Waveform Data}
Earthquakes originate from the abrupt release of accumulated elastic strain in 
the lithosphere, generating seismic waves that propagate through the Earth's 
interior and along its surface. These waves encode information about the source 
process, subsurface structure, and propagation path, making them fundamental 
for earthquake detection and characterization. Seismic waveform data consist of 
continuous time series recordings of ground motion captured by seismometers 
deployed at various locations. Each waveform represents the ground 
displacement, velocity, or acceleration as a function of time. The waveforms 
are typically sampled at high frequencies (e.g., 100 Hz or higher) to capture 
the rapid variations associated with seismic events. The data are stored in 
standard formats such as SEED or SAC, which include metadata about the 
recording station, instrument response, and event information.

\subsection{Types of Seismic Waves}
Seismic waves can be broadly classified into \textbf{body waves} and 
\textbf{surface waves}, each characterized by distinct propagation mechanisms 
and velocities.
\begin{itemize}
  \item Body waves are seismic waves that travel through the Earth's interior 
        and are the fastest seismic waves. Body waves are further divided into 
        two types: P waves and S waves.
        \begin{itemize}
          \item P waves or Primary waves are the fastest seismic waves and 
                travel through solids, liquids, and gases. P waves are also 
                known as compressional (longitudinal) waves because they 
                compress and expand the material through which they travel.
          \item S waves are secondary waves that are slower than P waves and 
                travel only through solids. S waves are also known as shear 
                (transversal) waves because they move the material through 
                which they travel from side to side.
        \end{itemize}
  \item Surface waves are seismic waves that travel along the Earth's surface 
        and are the slowest seismic waves. Surface waves are further divided
        into two types: Love waves and Rayleigh waves.
\end{itemize}
The velocities of these phases, strictly depend on the properties of the 
material they are traveling through, in particular the rock's density, 
compressibility and shear modulus of the material. Variations in these 
properties result in measurable differences in travel-time patterns, waveform 
shapes, and frequency content, which are essential for inversion and event 
localization.

\subsection{Data Acquisition}
The seismic waveform data are recorded using a network of seismometers deployed 
across northeastern Italy. Each instrument has their own advantages and 
disadvantages, and is used for different purposes. Seismometers are instruments 
that measure the velocity of the ground motion, while accelerometers are 
instruments that measure the acceleration of the ground motion. Each sensor 
records the ground motion in three orthogonal components: vertical (Z), 
north-south (N), and east-west (E). The data are transmitted in real-time to a 
central data acquisition system, where they are stored and processed for 
earthquake detection and analysis. The waveforms are archived in standardized 
formats such as MiniSEED or SAC, which include metadata about the recording 
station, instrument response, and event information. We utilize publicly 
available algorithms and tools such as ObsPy (\cite{ObsPy}) and SeisBench 
(\cite{SeisBench}) to access, process, and analyze the seismic waveform data.
ObsPy is an open-source Python library for seismology that provides a wide 
range of tools for processing and analyzing seismic data. ObsPy includes 
modules for downloading data from various data centers called \textbf{Mass 
Downloader}, filtering, resampling, and visualization. SeisBench is an 
open-source Python library that provides a standardized interface for accessing
and processing seismic waveform data. SeisBench includes a wide range of 
predefined datasets, as well as tools for data augmentation, preprocessing, and
model evaluation.

\section{Datasets}
\subsection{Case Study}
In this study, we use continuous waveforms recorded by 147 \ac{SMINO} and 
surrounding seismic stations located at the North-Eastern Italy region (Fig. 
\ref{fig:AdriaArray}). The \ac{SMINO} is managed by OGS also on behalf of the 
\ac{FVG} and Veneto Regions. The stations are installed in \ac{FVG}, Veneto 
(in most cases), Emilia Romagna and Lombardy. The data recorded by all the 
stations are acquired in real time at the OGS Centre for Seismological Research 
in Udine. The network also provides data to the national seismic surveillance 
system, with real-time data exchange with the Civil Protection Department and 
the National Institute of Geophysics and Volcanology (\ac{INGV}). Moreover, in 
order to improve the quality of localizations and magnitude estimation in 
border regions, the network exchanges real-time data with the seismic networks 
of Austria, Slovenia, Switzerland, and the Autonomous Provinces of Trento and 
Bolzano. The waveforms are recorded in three components (Z, N, and E, which 
represent the vertical, north-south, and east-west components of the seismic 
wave respectively), and different sampling rate (100Hz, 200Hz the most 
represented). Data are stored in MSEED format and daily files for each seismic 
station and component. The analyzed timeframe is from 20/03/2024 to 20/06/2024.
During this period, a total of 428 earthquakes were recorded by the network, 
with 7152 manually reviewed picks (P and S phases) available in the catalog. 
The dataset includes a diverse range of seismic events with magnitudes ranging 
from 0 to above 4.5, with depths ranging from shallow (a few kilometers) to 
intermediate depths (up to 70 km), providing a comprehensive representation of 
the seismic activity in northeastern Italy.

  \printbibliography
\end{document}